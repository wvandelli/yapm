\documentclass{beamer}
\usepackage{verbatim}
\usetheme{Boadilla}
\title% (optional, only for long titles)
{Partition configuration maker}
\author[Argyris Zardilis] % (optional, for multiple authors)
{Argyris Zardilis \inst{}}
\institute[Uni Cambridge \& CERN] % (optional)
{
  \inst{}%
  University of Cambridge \& CERN
}
\date[DAQ/HLT meeting, 15/08/2013] % (optional)
{DAQ/HLT meeting, 15 August 2013}



\begin{document}

\frame{\titlepage}

\begin{frame}
  \frametitle{Motivation}
  \begin{itemize}
    \item Large amount of information in a partition, hard to create its configuration manually \\
      \textbf{Solution}: Use a tool that automates all or part of the process

    \item Existing tool depends heavily on system design and schema. After new DF design can't do the
      job anymore.

    \item Last Technical Run: need to create partition of different flavours to test different components
      of the system and their combinations.

    \item Manual partition configuration generation laborious and time-consuming

    \item Limited amount of time in a TR setting to perform all necessary tests
  \end{itemize}
  
\end{frame}


\begin{frame}
  \frametitle{Solution}
  \begin{itemize}
    \item Created a tool following the new system design that automates the process

    \item Started as a temporary solution for TR2 but since grew to a more complete
      and configurable command-line tool akin to the older one.

    \item By no means as complete or comprehensive but covers basic use cases

    \item Set of python scripts to create basic segments or complete partitions

    \item Build on top of config and Python DAL package to get access to base classes defined in OKS schema and
      OKS ConfigObject(for classes not in standard schema bound to Python DAL)

    \item Also uses pm.project from the old package to get a convenient handle to the config db

  \end{itemize}

\end{frame}

\begin{frame}
 \frametitle{Solution schematically}
 Include a diagram of the modules
 
\end{frame}

\begin{frame}
 \frametitle{Capabilities}
 \framesubtitle{and incapabilities}
 \begin{itemize}
   \item Create localhost or multihost partitions in testbed/P1
     \begin{itemize}
       \item farm description loaded from user provided python dictionary
     \end{itemize}
   \item Create standard partitions: only DCMs, only HLTPUs
     \begin{itemize}
       \item customisable through command line parameters
     \end{itemize}

   \item Create standalone, pluggable HLTSV segment
     
   \item Doesn't handle ROS segment generation yet

   \item Also includes configuration for standard monitoring applications
     
   \item Other configurable parameters: extra includes, data networks, repository root

 \end{itemize}
 
\end{frame}


\begin{frame}
  \frametitle{Example usage}
  Create a DCM only partition with a provided python file 'farm\_gen' containing farm
  description with partition name 'az\_test' and repository root my home directory:
  \begin{itemize}
    \item \texttt{tdaq\_python pm\_evo.py -p az\_test -f farm\_gen --dcm-only -r tbed/user/azardili/installed}

  \end{itemize}

    Create a DCM/HLTPU partition with a provided python file 'farm\_gen' containing farm
  description with partition name 'az\_test' with PuDummy.data.xml as extra include:
  \begin{itemize}
    \item \texttt{tdaq\_python pm\_evo.py -p az\_test -f farm\_gen -I PuDummy.data.xml}

  \end{itemize}
  


\end{frame}

\begin{frame}
  \frametitle{Conclusions}
  \framesubtitle{and Future Work}
  \begin{itemize}
    \item More crude version proved useful for some use-cases in TR2.

    \item Short term plans: Use this more polished version more extensively
      in next TR or for tests in testbed
      \begin{itemize}
        \item Bonus: can be used by anyone now and it doesn't require knowledge of the
          code anymore!
      \end{itemize}

    \item Longer term plans: Build upon this based on usage experience and input from
      more people so that it becomes a complete and mature tool
      \begin{itemize}

        \item Discover use-cases and extra requirements
       \end{itemize}

    \item Current version lives in a private git repo on my public afs
      partition which you can clone:
      \begin{itemize}
        \item \texttt{/afs/cern.ch/user/a/azardili/public/partition\_maker}
      \end{itemize}
  \end{itemize}
\end{frame}


\end{document}
